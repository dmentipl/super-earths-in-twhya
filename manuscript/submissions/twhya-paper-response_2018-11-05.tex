\documentclass{article}

\usepackage{color} \usepackage{parskip} \usepackage{geometry}

\begin{document}

{\color{blue}
In addition to addressing the reviewer's comments below, we removed one panel
from Figure 2, showing the highest mass model. We also removed reference to that
model in Section 2.3, the final paragraph of Section 3, and the first paragraph
of Section 3.1. Given the results of the work we think the focus should be on
the lower mass models shown in Figure 2 and discussed in the text.
}

Reviewer's Comments:

The authors show that 2 terrestrial mass planets can explain the gaps observed
in the inner disc regions of TW Hya from dust thermal emission, and at the same
time a Saturn mass planet can explain the gap observed in the outer disc region
from scattered light and CO observations.  The work has improved considerably
after having ran extra simulations with 1 mm dust grains. The main points raised
in the previous referee report have been addressed.  I recommend publication on
MNRAS after minor revision.  Here is a list of points that should be explained
more thoroughly:

- Sec. 1 end of 4th paragraph: ``Grains with St~1 experience\ldots'' This phrase is
misleading. It seems like grains with St~1 are forming rings. It should be
specified that this happens only when a pressure gradient is present, caused by
one of the mechanisms specified in the following paragraph.

{\color{blue}
Done. Sentence amended accordingly.
}

- Sec. 2.2 first paragraph: 59.5 pc, not au

{\color{blue}
Done.
}

- Sec. 2.2 second paragraph: it is better not to state R\_in = 1 au, in order
not to confuse the value of R\_in in the definition of the Surface density given
later. I would just state: ``Extrapolating to 1 au implies…”

{\color{blue}
Amended, as suggested.
}

- Sec. 2.2 last paragraph: what is the point of discussing the dust/gas mass
ratio for the PHANTOM simulations where there is no back reaction?

{\color{blue}
Done. We added a sentence to Sec 2.1 clarifying that we do include back reaction
but only from individually modelled grain sizes.
}

- Sec. 2.3: the first planet is placed in the region of the gas surface density
profile where it is smoothed, so the density is decreasing. Isn't this affecting
the gap profile?

{\color{blue}
We agree. We added a caveat in the first paragraph of the results section.
}

- Sec. 2.3: it is not correct to refer at the 7:2 resonance as a MMR, since this
is the case only when, in a m:n resonance, m=n+1. It should be changed to only
resonance.

{\color{blue}
Done. We changed the sentence to simply give the period ratio and point out that
it is close to the peak in the distribution of period ratios of Kepler planet
pairs. We also moved this sentence to the introduction (in the final paragraph).
}

- Sec. 2.4 second-last paragraph: We then we add\ldots (one ``we'' too much)

{\color{blue}
Fixed.
}

- Sec. 4 first paragraph: I would refer to Fig. 2 when discussing the dust
thermal emission

{\color{blue}
Done.
}

- Sec. 4 6th paragraph: The viscosity is constrained from observations but the
authors nevertheless choose the lower limit of the derived range.

{\color{blue}
   Done. The viscosity constraint is a range of upper limits. We are
below the upper limit. We added text to make this clearer.
}

\end{document}
