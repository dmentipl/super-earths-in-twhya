\documentclass{article}

\usepackage{color}
\usepackage{parskip}
\usepackage{geometry}

\begin{document}

I have read with the great interest the letter ``Super-Earths in the TW Hya
disc'' by Mentiplay et al.  I think their work is worth publication on MNRAS
after a moderate revision.  The authors obtain mass constraints on the potential
planetary objects responsible for the gaps and (hint of a) spiral structure
observed in the TW Hya disc.  They compared their numerical simulations with a
range of observational data, from the dust continuum emission from ALMA, to the
polarized scatter light from SPHERE, and CO emission maps.  The weak points that
need to be addressed are:

1.\ the very short integration time (in particular for the outer planet, 16
orbits), and how this can influence the estimated planetary mass.  Especially
using a very low viscosity, the time-scale needed to open up a gap in the gas
disc can be of several tens of orbits. This can alter the conclusions, since
also a smaller planet than the ones considered can create those features
(especially at 94 au). It would be nice to show a time-evolution of the gap
structure (or of Fig. 4) to understand the importance of this factor.

{\color{blue}
   We extended our calculations. For the gas-only calculations where we look at
   the gap at 94~au we run for 100 orbits at this location. For the dust+gas
   calculations we ran for 100 orbits at the location of the 41~au planet.

   We added a fifth figure showing the gap profile in surface density as it
   varies in time.
}

2.\ the choice of only one intermediate dust size to model the entire dust
population.  It is reasonable to assume grains smaller than 1 micrometer
completely coupled to the gas, and interpolate between the behavior of the gas
and the modelled dust for intermediate sizes (up to St$\sim$0.3). It is not
clear how bigger dust particles (in particular around St$\sim$1) are treated and
how a simple interpolation can catch their behavior.

{\color{blue}
   We ran extra calculations with 1~mm dust grains (with St~$\sim$~3) to
   complement the 100~$\mu$m calculations. We then used both these grain size
   distributions as input for the radiative transfer calculation. All figures
   have been updated accordingly.
}

3.\ the constraint on the disc mass, although is strongly highlighted, is based
only on one model that is able to recreate the CO emission maps (an esploration
of the disc mass dependence is missing), and the observed turbulence (also model
dependent).

{\color{blue}
   We have softened the language on constraining the disc mass, and added
   further discussion on this.
}

Other minor issues:

- a strong argument in favour of a planetary system causing the observed gaps is
the possible resonance trapping migration of the planets. In particular the
inner two can be close to a 7:3 MMR\@. This potential aspect is not touched.

{\color{blue}
   We have added a sentence discussing this.
}

- they should estrapolate the disc mass up to the star, and not up to 10 au
(sec. 2.2), in order to have a realistic value to compare with obervations.

{\color{blue}
   We have addressed this by stating magnitude of the correction (1--10\%
   depending on prescription, i.e.\ power law, smoothed inner edge).
}

- an additional constraint on the outer planet can be inferred by the lack of
dust continuum emission outside its location. This is potentially an indication
that it has not reached the pebble isolation mass and dust particles can cross
its location. They can use the prescription by Bitsch et al. (2018) to put an
upper limit on the outer planet based on that.

{\color{blue}
   We have added discussion of pebble mass suggesting that radial drift may have
   occurred prior to the planet reaching its current mass.
}

- the strong acretion rate onto the planets (together with planet migration) can
prevent a stable state to be reached. What is the final mass of the outer
planet? How can this affect the mass prediction?

{\color{blue}
   We have added a paragraph giving the mass accreted onto each planet for each
   model.
}

- how was the check on the temperature at the innermost planet location made in
order to say that the effect of not having the disk inside 10 au was negligible?

{\color{blue}
   We have addressed this by explaining how we checked that the temperature
   beyond 20 au is accurately calculated in the radiative transfer model.
}

\end{document}
